\subsection{Fitting routine}
In order to extract scaling exponents from numerical data, we employed a \textit{Levenberg-Marquardt}-algorithm following the implementation given by \cite{Press92}. The precise fitting routine is as follows: For the scaling in time, we determine a fitting range $[t_{\text{min}},t_{\text{max}}]$ by inspecting the lowest moment $n_c$ of visited sites that scales non-logarithmically. This order is given by the dimension of the lattice via $n_c = \max(3-d,0)$. That obtained fitting range is then also used for all higher moments. For the scaling in system size, we use the full range of available data points.

Once the ranges are obtained, we first fit a pure power-law function $y_1(x) = A_1\,x^{B_1}$ against the data. Next, we use the obtained results as initial guesses to fit a corrected power-law $y_2(x) = A_2 \, x^{B_2} + C_2 \, x^{B_2 - \frac{1}{2}}$. The choice of the corrected scaling exponent is motivated by [ref?]. We then assign a quality of fit 
\begin{align}Q = \Gamma_{\frac{N-2}{2}}\left(\frac{\chi^2}{2}\right)
	\label{eq:goodness_of_fit}
\end{align}
where $\Gamma_m(x)$ is the incomplete gamma-function, $N$ the number of data-points, and $\chi^2$ the squared displacement of the data. This formula is taken from \cite{Press92}. We then refer to $B_2$ as the scaling exponent and $Q$ as its quality. A fit is rejected, if $Q\leq 0.1$. % Is that true? We always get 0.999 anyway

\subsection{Discussion}
For the goodness of fit (cf. Eq. \eqref{eq:goodness_of_fit}) to be valid, we assume that datapoints are normal-distributed around the fitting function and stochastically independent. Both assumptions, however, are incorrect; the trace-distribution is skewed (cf. [ref])
% Here we could link to the trace-histogram
and the trace and its moments are monotonous functions in time, hence subsequent datapoints are not independent.
% This should really sound less negative

%%%%%%%%%%%%%%%%%%%%%%%%
% BibTex-file
@book{Press:1992,
	author = {Press, William H. and Teukolsky, Saul A. and Vetterling, William T. and Flannery, Brian P.},
	title = {Numerical Recipes in C (2Nd Ed.): The Art of Scientific Computing},
	year = {1992},
	isbn = {0-521-43108-5},
	publisher = {Cambridge University Press},
	address = {New York, NY, USA},
      } 
